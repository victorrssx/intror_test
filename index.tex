% Options for packages loaded elsewhere
\PassOptionsToPackage{unicode}{hyperref}
\PassOptionsToPackage{hyphens}{url}
\PassOptionsToPackage{dvipsnames,svgnames,x11names}{xcolor}
%
\documentclass[
  letterpaper,
  DIV=11,
  numbers=noendperiod]{scrreprt}

\usepackage{amsmath,amssymb}
\usepackage{iftex}
\ifPDFTeX
  \usepackage[T1]{fontenc}
  \usepackage[utf8]{inputenc}
  \usepackage{textcomp} % provide euro and other symbols
\else % if luatex or xetex
  \usepackage{unicode-math}
  \defaultfontfeatures{Scale=MatchLowercase}
  \defaultfontfeatures[\rmfamily]{Ligatures=TeX,Scale=1}
\fi
\usepackage{lmodern}
\ifPDFTeX\else  
    % xetex/luatex font selection
\fi
% Use upquote if available, for straight quotes in verbatim environments
\IfFileExists{upquote.sty}{\usepackage{upquote}}{}
\IfFileExists{microtype.sty}{% use microtype if available
  \usepackage[]{microtype}
  \UseMicrotypeSet[protrusion]{basicmath} % disable protrusion for tt fonts
}{}
\makeatletter
\@ifundefined{KOMAClassName}{% if non-KOMA class
  \IfFileExists{parskip.sty}{%
    \usepackage{parskip}
  }{% else
    \setlength{\parindent}{0pt}
    \setlength{\parskip}{6pt plus 2pt minus 1pt}}
}{% if KOMA class
  \KOMAoptions{parskip=half}}
\makeatother
\usepackage{xcolor}
\setlength{\emergencystretch}{3em} % prevent overfull lines
\setcounter{secnumdepth}{5}
% Make \paragraph and \subparagraph free-standing
\ifx\paragraph\undefined\else
  \let\oldparagraph\paragraph
  \renewcommand{\paragraph}[1]{\oldparagraph{#1}\mbox{}}
\fi
\ifx\subparagraph\undefined\else
  \let\oldsubparagraph\subparagraph
  \renewcommand{\subparagraph}[1]{\oldsubparagraph{#1}\mbox{}}
\fi

\usepackage{color}
\usepackage{fancyvrb}
\newcommand{\VerbBar}{|}
\newcommand{\VERB}{\Verb[commandchars=\\\{\}]}
\DefineVerbatimEnvironment{Highlighting}{Verbatim}{commandchars=\\\{\}}
% Add ',fontsize=\small' for more characters per line
\usepackage{framed}
\definecolor{shadecolor}{RGB}{241,243,245}
\newenvironment{Shaded}{\begin{snugshade}}{\end{snugshade}}
\newcommand{\AlertTok}[1]{\textcolor[rgb]{0.68,0.00,0.00}{#1}}
\newcommand{\AnnotationTok}[1]{\textcolor[rgb]{0.37,0.37,0.37}{#1}}
\newcommand{\AttributeTok}[1]{\textcolor[rgb]{0.40,0.45,0.13}{#1}}
\newcommand{\BaseNTok}[1]{\textcolor[rgb]{0.68,0.00,0.00}{#1}}
\newcommand{\BuiltInTok}[1]{\textcolor[rgb]{0.00,0.23,0.31}{#1}}
\newcommand{\CharTok}[1]{\textcolor[rgb]{0.13,0.47,0.30}{#1}}
\newcommand{\CommentTok}[1]{\textcolor[rgb]{0.37,0.37,0.37}{#1}}
\newcommand{\CommentVarTok}[1]{\textcolor[rgb]{0.37,0.37,0.37}{\textit{#1}}}
\newcommand{\ConstantTok}[1]{\textcolor[rgb]{0.56,0.35,0.01}{#1}}
\newcommand{\ControlFlowTok}[1]{\textcolor[rgb]{0.00,0.23,0.31}{#1}}
\newcommand{\DataTypeTok}[1]{\textcolor[rgb]{0.68,0.00,0.00}{#1}}
\newcommand{\DecValTok}[1]{\textcolor[rgb]{0.68,0.00,0.00}{#1}}
\newcommand{\DocumentationTok}[1]{\textcolor[rgb]{0.37,0.37,0.37}{\textit{#1}}}
\newcommand{\ErrorTok}[1]{\textcolor[rgb]{0.68,0.00,0.00}{#1}}
\newcommand{\ExtensionTok}[1]{\textcolor[rgb]{0.00,0.23,0.31}{#1}}
\newcommand{\FloatTok}[1]{\textcolor[rgb]{0.68,0.00,0.00}{#1}}
\newcommand{\FunctionTok}[1]{\textcolor[rgb]{0.28,0.35,0.67}{#1}}
\newcommand{\ImportTok}[1]{\textcolor[rgb]{0.00,0.46,0.62}{#1}}
\newcommand{\InformationTok}[1]{\textcolor[rgb]{0.37,0.37,0.37}{#1}}
\newcommand{\KeywordTok}[1]{\textcolor[rgb]{0.00,0.23,0.31}{#1}}
\newcommand{\NormalTok}[1]{\textcolor[rgb]{0.00,0.23,0.31}{#1}}
\newcommand{\OperatorTok}[1]{\textcolor[rgb]{0.37,0.37,0.37}{#1}}
\newcommand{\OtherTok}[1]{\textcolor[rgb]{0.00,0.23,0.31}{#1}}
\newcommand{\PreprocessorTok}[1]{\textcolor[rgb]{0.68,0.00,0.00}{#1}}
\newcommand{\RegionMarkerTok}[1]{\textcolor[rgb]{0.00,0.23,0.31}{#1}}
\newcommand{\SpecialCharTok}[1]{\textcolor[rgb]{0.37,0.37,0.37}{#1}}
\newcommand{\SpecialStringTok}[1]{\textcolor[rgb]{0.13,0.47,0.30}{#1}}
\newcommand{\StringTok}[1]{\textcolor[rgb]{0.13,0.47,0.30}{#1}}
\newcommand{\VariableTok}[1]{\textcolor[rgb]{0.07,0.07,0.07}{#1}}
\newcommand{\VerbatimStringTok}[1]{\textcolor[rgb]{0.13,0.47,0.30}{#1}}
\newcommand{\WarningTok}[1]{\textcolor[rgb]{0.37,0.37,0.37}{\textit{#1}}}

\providecommand{\tightlist}{%
  \setlength{\itemsep}{0pt}\setlength{\parskip}{0pt}}\usepackage{longtable,booktabs,array}
\usepackage{calc} % for calculating minipage widths
% Correct order of tables after \paragraph or \subparagraph
\usepackage{etoolbox}
\makeatletter
\patchcmd\longtable{\par}{\if@noskipsec\mbox{}\fi\par}{}{}
\makeatother
% Allow footnotes in longtable head/foot
\IfFileExists{footnotehyper.sty}{\usepackage{footnotehyper}}{\usepackage{footnote}}
\makesavenoteenv{longtable}
\usepackage{graphicx}
\makeatletter
\def\maxwidth{\ifdim\Gin@nat@width>\linewidth\linewidth\else\Gin@nat@width\fi}
\def\maxheight{\ifdim\Gin@nat@height>\textheight\textheight\else\Gin@nat@height\fi}
\makeatother
% Scale images if necessary, so that they will not overflow the page
% margins by default, and it is still possible to overwrite the defaults
% using explicit options in \includegraphics[width, height, ...]{}
\setkeys{Gin}{width=\maxwidth,height=\maxheight,keepaspectratio}
% Set default figure placement to htbp
\makeatletter
\def\fps@figure{htbp}
\makeatother

\KOMAoption{captions}{tableheading}
\makeatletter
\@ifpackageloaded{tcolorbox}{}{\usepackage[skins,breakable]{tcolorbox}}
\@ifpackageloaded{fontawesome5}{}{\usepackage{fontawesome5}}
\definecolor{quarto-callout-color}{HTML}{909090}
\definecolor{quarto-callout-note-color}{HTML}{0758E5}
\definecolor{quarto-callout-important-color}{HTML}{CC1914}
\definecolor{quarto-callout-warning-color}{HTML}{EB9113}
\definecolor{quarto-callout-tip-color}{HTML}{00A047}
\definecolor{quarto-callout-caution-color}{HTML}{FC5300}
\definecolor{quarto-callout-color-frame}{HTML}{acacac}
\definecolor{quarto-callout-note-color-frame}{HTML}{4582ec}
\definecolor{quarto-callout-important-color-frame}{HTML}{d9534f}
\definecolor{quarto-callout-warning-color-frame}{HTML}{f0ad4e}
\definecolor{quarto-callout-tip-color-frame}{HTML}{02b875}
\definecolor{quarto-callout-caution-color-frame}{HTML}{fd7e14}
\makeatother
\makeatletter
\@ifpackageloaded{bookmark}{}{\usepackage{bookmark}}
\makeatother
\makeatletter
\@ifpackageloaded{caption}{}{\usepackage{caption}}
\AtBeginDocument{%
\ifdefined\contentsname
  \renewcommand*\contentsname{Índice}
\else
  \newcommand\contentsname{Índice}
\fi
\ifdefined\listfigurename
  \renewcommand*\listfigurename{Lista de Figuras}
\else
  \newcommand\listfigurename{Lista de Figuras}
\fi
\ifdefined\listtablename
  \renewcommand*\listtablename{Lista de Tabelas}
\else
  \newcommand\listtablename{Lista de Tabelas}
\fi
\ifdefined\figurename
  \renewcommand*\figurename{Figura}
\else
  \newcommand\figurename{Figura}
\fi
\ifdefined\tablename
  \renewcommand*\tablename{Tabela}
\else
  \newcommand\tablename{Tabela}
\fi
}
\@ifpackageloaded{float}{}{\usepackage{float}}
\floatstyle{ruled}
\@ifundefined{c@chapter}{\newfloat{codelisting}{h}{lop}}{\newfloat{codelisting}{h}{lop}[chapter]}
\floatname{codelisting}{Listagem}
\newcommand*\listoflistings{\listof{codelisting}{Lista de Listagens}}
\makeatother
\makeatletter
\makeatother
\makeatletter
\@ifpackageloaded{caption}{}{\usepackage{caption}}
\@ifpackageloaded{subcaption}{}{\usepackage{subcaption}}
\makeatother
\ifLuaTeX
\usepackage[bidi=basic]{babel}
\else
\usepackage[bidi=default]{babel}
\fi
\babelprovide[main,import]{portuguese}
% get rid of language-specific shorthands (see #6817):
\let\LanguageShortHands\languageshorthands
\def\languageshorthands#1{}
\ifLuaTeX
  \usepackage{selnolig}  % disable illegal ligatures
\fi
\usepackage{bookmark}

\IfFileExists{xurl.sty}{\usepackage{xurl}}{} % add URL line breaks if available
\urlstyle{same} % disable monospaced font for URLs
\hypersetup{
  pdftitle={R e RStudio para Iniciantes},
  pdfauthor={GPEQ/UFRJ},
  pdflang={pt},
  colorlinks=true,
  linkcolor={blue},
  filecolor={Maroon},
  citecolor={Blue},
  urlcolor={Blue},
  pdfcreator={LaTeX via pandoc}}

\title{R e RStudio para Iniciantes}
\usepackage{etoolbox}
\makeatletter
\providecommand{\subtitle}[1]{% add subtitle to \maketitle
  \apptocmd{\@title}{\par {\large #1 \par}}{}{}
}
\makeatother
\subtitle{Material de Apoio para Cursos Quantitativos do Instituto de
Economia da Universidade Federal do Rio de Janeiro (IE/UFRJ)}
\author{GPEQ/UFRJ}
\date{2024-03-27}

\begin{document}
\maketitle

\renewcommand*\contentsname{Índice}
{
\hypersetup{linkcolor=}
\setcounter{tocdepth}{2}
\tableofcontents
}
\bookmarksetup{startatroot}

\chapter{Objects}\label{objects}

{[}\ldots{]}

\subsection{Tipo \& Forma}\label{tipo-forma}

Vamos nos aprofundar um pouco mais. Ao lidar formalmente com dados,
\textbf{devemos ter mente que eles são compostos por uma ou mais
variáveis e seus valores}. \emph{Uma variável é uma dimensão ou
propriedade que descreve uma unidade de observação} (por exemplo, uma
pessoa) e normalmente pode assumir valores diferentes. Por outro lado,
os \emph{valores são as instâncias concretas que uma variável atribui a
cada unidade de observação e são ainda caracterizados por seu intervalo}
(por exemplo, valores categóricos versus valores contínuos) \emph{e seu
tipo} (por exemplo, valores lógicos, numéricos ou de caracteres).
Estaremos interessados no \emph{tipo} dos dados. A
Tabela~\ref{tbl-data-types} apresenta os que podem aparecer com maior
frequência.

\begin{longtable}[]{@{}
  >{\centering\arraybackslash}p{(\columnwidth - 4\tabcolsep) * \real{0.1935}}
  >{\centering\arraybackslash}p{(\columnwidth - 4\tabcolsep) * \real{0.6129}}
  >{\centering\arraybackslash}p{(\columnwidth - 4\tabcolsep) * \real{0.1935}}@{}}

\caption{\label{tbl-data-types}Tipos mais comuns de dados}

\tabularnewline

\toprule\noalign{}
\begin{minipage}[b]{\linewidth}\centering
Tipo
\end{minipage} & \begin{minipage}[b]{\linewidth}\centering
Serve para representar\ldots{}
\end{minipage} & \begin{minipage}[b]{\linewidth}\centering
Exemplo
\end{minipage} \\
\midrule\noalign{}
\endhead
\bottomrule\noalign{}
\endlastfoot
Númerico & números do tipo \emph{integer} (inteiro) ou \emph{double}
(reais) & 1, 3.2, 0.89 \\
Texto \emph{(string)} & caracteres (letras, palavras ou setenças) &
``Ana jogou bola'' \\
Lógico & valores verdade do tipo lógico (valores booleanos) & TRUE,
FALSE, NA \\
Tempo & datas e horas & 14/04/1999 \\

\end{longtable}

Voltando ao primeiro exemplo, uma pessoa pode ser descrita pelas
variáveis \emph{nome}, \emph{número de horas dormidas} e \emph{se dormiu
ou não mais de oito horas}. Os valores correspondentes a essas variáveis
seriam do tipo texto (por exemplo, ``Pedro''), numéricos (número de
horas) e lógicos (\texttt{TRUE} ou \texttt{FALSE}, definido em função do
tempo descansado\footnote{Se o número de horas que a pessoa descansou
  for maior do que 8, então a variável deverá apresentar valor igual a
  \texttt{TRUE} -- ou seja, é verdade que a pessoa dormiu mais de 8
  horas. Caso contrário, \texttt{FALSE}.}). \textbf{Note a diferença
entre \emph{dado} e \emph{valor}.} O número 10 é um valor, sem
significado. Por outro lado, \emph{``10 horas dormidas''} é um dado,
caracterizado pelo valor 10 e pela variável \emph{``horas dormidas''}.

Outro aspecto importante sobre os dados está em sua forma, ou seja, como
os dados podem ser organizados. A Tabela~\ref{tbl-data-shapes} apresenta
as formas mais comuns de organização.

\begin{longtable}[]{@{}
  >{\centering\arraybackslash}p{(\columnwidth - 4\tabcolsep) * \real{0.1758}}
  >{\centering\arraybackslash}p{(\columnwidth - 4\tabcolsep) * \real{0.5165}}
  >{\centering\arraybackslash}p{(\columnwidth - 4\tabcolsep) * \real{0.3077}}@{}}

\caption{\label{tbl-data-shapes}Formas pelas quais os dados podem ser
organizados}

\tabularnewline

\toprule\noalign{}
\begin{minipage}[b]{\linewidth}\centering
Formato
\end{minipage} & \begin{minipage}[b]{\linewidth}\centering
Os dados se apresentam como\ldots{}
\end{minipage} & \begin{minipage}[b]{\linewidth}\centering
Exemplo
\end{minipage} \\
\midrule\noalign{}
\endhead
\bottomrule\noalign{}
\endlastfoot
Escalar & elementos individuais & ``AB'', 4, TRUE \\
Retangular & dados organizados em \(i\) linhas e \(j\) colunas & Vetores
e Tabelas de Dados \\
Não-retangular & junção de uma ou mais estruturas de dados & Listas \\

\end{longtable}

{[}\ldots{]}

\bookmarksetup{startatroot}

\chapter{Primeiros passos}\label{primeiros-passos}

\begin{tcolorbox}[enhanced jigsaw, toprule=.15mm, breakable, left=2mm, opacityback=0, colback=white, rightrule=.15mm, leftrule=.75mm, arc=.35mm, colframe=quarto-callout-note-color-frame, bottomrule=.15mm]

Partes deste capítulo são baseadas na seção
\href{https://livro.curso-r.com/3-2-r-como-calculadora.html}{3.2 `R como
calculadora'} do livro \emph{Ciência de Dados em R}, feito pelo Curso-R.
De qualquer modo, eventuais erros são inteiramente de nossa
responsabilidade.

\end{tcolorbox}

Como vimos nos capítulos anteriores, o papel do \textbf{Console} no R é
interpretar os nossos comandos à luz da linguagem. Ele avalia o código
que o passamos e devolve a saída correspondente --- se tudo der certo
--- ou uma mensagem de erro --- se o seu código tiver algum problema.
Essa operação é chamada de \textbf{avaliar}, \textbf{executar} ou
\textbf{rodar} o código. Para que seu código seja executado diretamente
no Console, escreva-o e, na sequência, aperte \texttt{Enter}. A outra
forma de executar uma expressão é escrever o código em um \emph{script}
no \textbf{Editor}, deixar o cursor em cima da linha e usar o atalho
\texttt{Ctrl\ +\ Enter}. Assim, o comando é enviado para o Console, onde
é diretamente executado.

Nesse capítulo, você \emph{rodará} suas primeiras linhas de código com
intuito de realizar operações aritméticas como \emph{adição},
\emph{subtração}, \emph{multiplicação} e \emph{divisão}, além de
comparações lógicas simples. O objetivo aqui não é te ensinar matemática
básica, mas te preparar para a execução de linhas de código mais
avançadas. É a forma mais fácil de um iniciante ganhar familiaridade e
experiência prática com o R.

\section{Operadores Aritméticos}\label{operadores-aritmuxe9ticos}

De agora em diante, cada região sombreada de cinza representa código, ao
passo que seu resultado estará exposto logo na sequência. Vamos começar
com um exemplo simples:

\begin{Shaded}
\begin{Highlighting}[]
\DecValTok{1} \SpecialCharTok{+} \DecValTok{1}
\end{Highlighting}
\end{Shaded}

\begin{verbatim}
[1] 2
\end{verbatim}

Nesse caso, o nosso comando foi o código \texttt{1\ +\ 1} e a saída foi
o valor \texttt{2}. Como você pode reproduzir esse comando no RStudio?
Inicialmente, copie o que está escrito acima ao clicar no símbolo de
prancheta no canto superior direito da região sombreada. Na sequência,
cole no Editor de Código e aperte \texttt{Ctrl\ +\ Enter} (ou então no
Console, pressionando apenas \texttt{Enter}). Observe abaixo!

\begin{center}
\includegraphics{images/1mais1.gif}
\end{center}

Tente agora jogar no Console a expressão:
\texttt{2\ *\ 2\ -\ (4\ +\ 4)\ /\ 2}. Deu zero? Pronto! Você já é capaz
de pedir ao R para fazer \emph{qualquer uma das quatro operações
aritméticas básicas}. Repare que as operações e suas precedências são
mantidas como na matemática, ou seja, divisão e multiplicação são
calculadas antes da adição e subtração, além de os parênteses ditarem a
ordem na qual serão realizadas. A seguir, apresentamos a
Tabela~\ref{tbl-ope-mat} resumindo como fazer as principais operações no
R.

\begin{longtable}[]{@{}cccc@{}}

\caption{\label{tbl-ope-mat}Operadores matemáticos do R}

\tabularnewline

\toprule\noalign{}
Operação & Operador & Exemplo & Resultado \\
\midrule\noalign{}
\endhead
\bottomrule\noalign{}
\endlastfoot
Adição & + & 1 + 1 & 2.00 \\
Subtração & - & 4 - 2 & 2.00 \\
Multiplicação & * & 2 * 3 & 6.00 \\
Divisão & / & 5 / 3 & 1.67 \\
Potenciação & \^{} & 4 \^{} 2 & 16.00 \\
Resto da Divisão & \%\% & 5 \%\% 3 & 2.00 \\
Parte Inteira da Divisão & \%/\% & 5 \%/\% 3 & 1.00 \\

\end{longtable}

\section{Operadores Lógicos}\label{operadores-luxf3gicos}

O R permite também testar comparações lógicas. Os valores lógicos
básicos em R são \texttt{TRUE} (ou apenas \texttt{T}) e \texttt{FALSE}
(ou apenas \texttt{F}). Por exemplo, podemos pedir ao R que nos diga se
é verdadeiro que 5 é menor do que 3. Como a resposta é obviamente
negativa, ele retornará \texttt{FALSE}, nos dizendo que a proposição que
fizemos é falsa.

\begin{Shaded}
\begin{Highlighting}[]
\DecValTok{5} \SpecialCharTok{\textless{}} \DecValTok{3}
\end{Highlighting}
\end{Shaded}

\begin{verbatim}
[1] FALSE
\end{verbatim}

Abaixo, introduzimos a Tabela~\ref{tbl-ope-log} com outros operadores
lógicos da linguagem.

\begin{longtable}[]{@{}
  >{\centering\arraybackslash}p{(\columnwidth - 6\tabcolsep) * \real{0.2151}}
  >{\centering\arraybackslash}p{(\columnwidth - 6\tabcolsep) * \real{0.1075}}
  >{\centering\arraybackslash}p{(\columnwidth - 6\tabcolsep) * \real{0.5591}}
  >{\centering\arraybackslash}p{(\columnwidth - 6\tabcolsep) * \real{0.1183}}@{}}

\caption{\label{tbl-ope-log}Operadores lógicos do R}

\tabularnewline

\toprule\noalign{}
\begin{minipage}[b]{\linewidth}\centering
Operação
\end{minipage} & \begin{minipage}[b]{\linewidth}\centering
Operador
\end{minipage} & \begin{minipage}[b]{\linewidth}\centering
Exemplo
\end{minipage} & \begin{minipage}[b]{\linewidth}\centering
Resultado
\end{minipage} \\
\midrule\noalign{}
\endhead
\bottomrule\noalign{}
\endlastfoot
Maior que & \textgreater{} & 2 \textgreater{} 1 & TRUE \\
Maior ou igual que & \textgreater= & 2 \textgreater= 2 & TRUE \\
Menor que & \textless{} & 2 \textless{} 3 & TRUE \\
Menor ou igual que & \textless= & 5 =\textless{} 3 & FALSE \\
Igual à & == & 4 == 4 & TRUE \\
Diferente de & != & 5 != 3 & TRUE \\
x \textbf{e} y & \& & x \textless- c(1, 4, NA, 8) x{[}!is.na(x) \& x
\textgreater{} 5{]} & 8 \\
x \textbf{ou} y & \textbar{} & x \textless- c(1, 4, NA, 8) x{[}!is.na(x)
\textbar{} x \textgreater{} 5{]} & 1, 4, 8 \\

\end{longtable}

\section{Possíveis complicações}\label{possuxedveis-complicauxe7uxf5es}

Se você digitar um comando incompleto, como \texttt{5\ +}, e apertar
\texttt{Enter}, o R mostrará um \texttt{+}, o que não tem nada a ver com
a adição da matemática. Isso significa que o R está esperando você
enviar \textbf{mais} algum código para completar o seu comando. Termine
o seu comando ou aperte \texttt{Esc} para recomeçar.

\begin{Shaded}
\begin{Highlighting}[]
\DecValTok{5} \SpecialCharTok{{-}}
\SpecialCharTok{+} 
\SpecialCharTok{+} \DecValTok{5}
\end{Highlighting}
\end{Shaded}

\begin{verbatim}
[1] 0
\end{verbatim}

Se você digitar um comando que o R não reconhece, ele retornará uma
mensagem de erro. \textbf{Não entre em pânico.} Ele só está te avisando
que não conseguiu interpretar o comando.

\begin{Shaded}
\begin{Highlighting}[]
\DecValTok{5}\NormalTok{ \% }\DecValTok{2}
\end{Highlighting}
\end{Shaded}

\begin{verbatim}
Error: <text>:1:3: unexpected input
1: 5 % 2
      ^
\end{verbatim}

Você pode digitar outro comando normalmente em seguida.

\begin{Shaded}
\begin{Highlighting}[]
\DecValTok{5} \SpecialCharTok{\^{}} \DecValTok{2}
\end{Highlighting}
\end{Shaded}

\begin{verbatim}
[1] 25
\end{verbatim}



\end{document}
